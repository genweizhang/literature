
\documentclass[aip,jcp,preprint,superscriptaddress,amsmath,amssymb]{revtex4-1}


\usepackage{graphicx}% Include figure files
\usepackage{mathrsfs}
\usepackage{dcolumn}% Align table columns on decimal point
\usepackage{bm}% bold math
\usepackage{rotating}
\usepackage{textcomp}
\usepackage{float}
\usepackage{subfig}
\usepackage{color}
\usepackage{gensymb}
\usepackage{xspace} % fcp: useful for text macros
\usepackage[normalem]{ulem} % fcp: strike-out text for fcprev

\newcommand{\yerev}[1]{\textcolor{red}{#1}}
\newcommand{\fcprev}[1]{\textcolor{red}{#1}}
\newcommand{\ie}[0]{\textit{i.e.},\xspace} 
\newcommand{\etal}[0]{\textit{et al.}\xspace} 

\begin{document}

\section{Mulliken Charges} 

The Mulliken charges are defined as 
\begin{equation}
Q_A = \sum_{\mu \in A} \sum_{\nu}   P^{\mu \nu} S_{\mu \nu} 
\end{equation}
Their full derivatives with respect to nuclear displacements are
\begin{equation}
Q_A ^{(x)} = \sum_{\mu \in A} \sum_{\nu}   P^{\mu \nu,(x)}  S_{\mu \nu}  + \sum_{\mu \in A} \sum_{\nu}   P^{\mu \nu} S_{\mu \nu}^{(x)} 
\end{equation}
where the second term is called the M-derivative.     
The first term depends on the density response, 
\begin{equation}
P^{(x)} =  - \frac{1}{2} CC^{\dagger} S^{(x)} P + C_v \Theta^{(x)} C_o^{\dagger} 
\end{equation}
which include an overlap derivative component and an orbital relaxation component.    So the nuclear gradient of Mulliken charges are:
\begin{eqnarray}
Q_A ^{(x)} & = & Q_{A,PS} ^{(x)} + Q_{A,PO} ^{(x)} + Q_{A,M} ^{(x)}         \\
Q_{A,PS}^{(x)} & = & - \frac{1}{2}  \sum_{\mu \in A} \sum_{\nu}  \left[  CC^{\dagger} S^{(x)} P  \right] ^{\mu \nu}  S_{\mu \nu}  \\
Q_{A,OR} ^{(x)} & = & \sum_{\mu \in A} \sum_{\nu}  \left[   C_v \Theta^{(x)} C_o^{\dagger}  \right] ^{\mu \nu}  S_{\mu \nu}  \\
Q_{A,M} ^{(x)}   & = & \sum_{\mu \in A} \sum_{\nu}   P^{\mu \nu} S_{\mu \nu}^{(x)} 
\end{eqnarray}

When Mulliken charges are employed in an implicit solvent model (I am not aware of any) or QM/MM model, 
the orbital relaxation component ($Q_{A,OR} ^{(x)}$) can be ignored in an energy gradient calculation, 
while the overlap derivative component ($Q_{A,PS}^{(x)}$) is automatically folded into the Pulay term 
through the energy weighted density matrix.   Only the M-derivatives ($Q_{A,M} ^{(x)}$) need to be added explicitly.  

If, for any reason, full gradient is needed, we can solve CPSCF equation for the 3*NAtoms orbital responses (3 for each atom).   
But we might be able to save a little bit time by solving for only ONE response for each atom.   
This can be done by rewritten as the orbital relaxation component of the gradient as
\begin{eqnarray}
Q_{A,PO} ^{(x)}  &  = &  \sum_{\mu \in A} \sum_{\nu} \sum_{ai} C_{\mu a}  \Theta^{(x)}_{ai} C_{\nu i}  S_{\mu \nu}  \nonumber \\  
& = &   \sum_{\mu \in A} \sum_{\nu} \sum_{ai} C_{\mu a}  \sum_{bj} \left[ \left( E^{\Theta \Theta}  \right)^{-1} \right]_{ai, bj} h^{(x)}_{bj}  C_{\nu i}  S_{\mu \nu}  \nonumber \\
& = &  \sum_{bj} h^{(x)}_{bj}  \sum_{ai} \left[ \left( E^{\Theta \Theta}  \right)^{-1} \right]_{ai, bj} \left[  \sum_{\mu \in A} \sum_{\nu} C_{\mu a}  S_{\mu \nu}   C_{\nu i} \right]  \nonumber \\
& = &  \sum_{bj}  h^{(x)}_{bj}  \sum_{ai} \left[ \left( E^{\Theta \Theta}  \right)^{-1} \right]_{ai, bj}  S_{A, ai}  \nonumber  \\
& = & \sum_{bj}  h^{(x)}_{bj} Z_{A, bj} 
\end{eqnarray}
where 
\begin{equation}
S_{A, ai}  = \sum_{\mu \in A} \sum_{\nu} C_{\mu a}  S_{\mu \nu}   C_{\nu i}
\end{equation}
is the overlap between a normal occupied orbital (i) and a truncated virtual orbital (a),  which has only contributions from basis functions on atom A.  
$Z_{A, bj}$ is the corresponding z vector, 
\begin{equation}
Z_{A, bj}  = \sum_{ai} \left[ \left( E^{\Theta \Theta}  \right)^{-1} \right]_{ai, bj}  S_{A, ai} 
\end{equation}

%\bibliographystyle{acs}




\end{document}
