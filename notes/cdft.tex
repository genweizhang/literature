
\documentclass[aip,jcp,preprint,superscriptaddress,amsmath,amssymb]{revtex4-1}


\usepackage{graphicx}% Include figure files
\usepackage{mathrsfs}
\usepackage{dcolumn}% Align table columns on decimal point
\usepackage{bm}% bold math
\usepackage{rotating}
\usepackage{textcomp}
\usepackage{float}
\usepackage{subfig}
\usepackage{color}
\usepackage{gensymb}
\usepackage{xspace} % fcp: useful for text macros
\usepackage[normalem]{ulem} % fcp: strike-out text for fcprev

\newcommand{\yerev}[1]{\textcolor{red}{#1}}
\newcommand{\fcprev}[1]{\textcolor{red}{#1}}
\newcommand{\ie}[0]{\textit{i.e.},\xspace} 
\newcommand{\etal}[0]{\textit{et al.}\xspace} 
\newcommand{\inlinemaketitle}{{\let\newpage\relax\maketitle}}

\begin{document}

\begin{center} 
\textbf{Estimation of the Bias Potential and Charge Transfer Energy in Constrained Density Functional Theory Calculations}  \\
Yuezhi Mao and Yihan Shao \\
(\today)  \\
\end{center} 



Given a weight matrix, $W_{p}$, using a population scheme, 
we can use its AO representation to obtain the unbiased population 
from the unbiased density matrix ($P^{(0)}$) 
\begin{equation}
Q_p^{0} = W_{p,ao} \cdot P^{(0)} + \mathrm{(nuclear~contribution)} 
\end{equation} 

We can convert the weight matrix into the MO representation (V $\times$ O),    
and compute its corresponding z vector, 
\begin{equation}
z_p = \left( E_{\Theta \Theta} \right)  ^{-1}  W_{p,mo} 
\end{equation}
From the z vector, we can compute the corresponding density matrix change (i.e. Fock build is incremented with one unit of the weight matrix)
\begin{equation}
\Delta P_p  = C_v z_p C_o^\dagger + C_o z_p^\dagger C_v^\dagger 
\end{equation}
and population change
\begin{equation}
\Delta Q_p = \Delta P_p \cdot W_{p,ao}  
\end{equation} 

From that, we can estimate the required bias potential for achieving a zero-charge fragment, 
\begin{equation}
\lambda_p = -  \frac{ Q_p^{0}  }  {\Delta Q_p  }
\end{equation} 
and the energy charge for this charge constraint is 
\begin{equation}
E ( \lambda_p ) = \frac{1}{2}~~ \lambda_p^2  ~~ (z_p \cdot W_{p,mo} ) 
\end{equation} 





\bibliography{tddft}
\bibliographystyle{acs}




\end{document}
