
\documentclass[aip,jcp,preprint,superscriptaddress,amsmath,amssymb]{revtex4-1}


\usepackage{graphicx}% Include figure files
\usepackage{mathrsfs}
\usepackage{dcolumn}% Align table columns on decimal point
\usepackage{bm}% bold math
\usepackage{rotating}
\usepackage{textcomp}
\usepackage{float}
\usepackage{subfig}
\usepackage{color}
\usepackage{gensymb}
\usepackage{xspace} % fcp: useful for text macros
\usepackage[normalem]{ulem} % fcp: strike-out text for fcprev

\newcommand{\yerev}[1]{\textcolor{red}{#1}}
\newcommand{\fcprev}[1]{\textcolor{red}{#1}}
\newcommand{\ie}[0]{\textit{i.e.},\xspace} 
\newcommand{\etal}[0]{\textit{et al.}\xspace} 

\begin{document}

\begin{center}
\textbf{Random Thoughts on Atomic-Site Electronegativity}  \\
(\today)
\end{center}
Given a molecular system with an external potential $v(\vec{r})$, 
Kohn-Sham density functional theory leads to a converged electron density, $\rho(\vec{r})$,
and the corresponding energy, $E[\rho(\vec{r})]$. 

Let us apply a small perturbation to the external potential,
\begin{equation}
v(\vec{r}) ~~~ \longrightarrow ~~~ v(\vec{r})  + \varepsilon ~\delta v(\vec{r}) 
\end{equation}
where $\varepsilon$ is a parameter that we can use to tune the strength of the perturbation.  
This perturbation will cause a change in the electron density
\begin{equation}
\rho(\vec{r}) ~~~ \longrightarrow ~~~ \rho(\vec{r})  + \varepsilon ~\delta \rho(\vec{r}) 
\end{equation}
and in the atomic populations 
\begin{equation}
N_A~~~ \longrightarrow ~~~ N_A + \varepsilon ~\delta N_A 
\end{equation}

Due to the conservation of the number of electrons, we have
\begin{equation}
\sum_A \delta N_A = 0 
\end{equation}
We can also  compute the energy derivative with respect to $\varepsilon$,
\begin{equation}
\frac { \partial E} { \partial \varepsilon}  = \sum_A  \left( \frac{\partial E} { \partial N_A } \right) ~ \delta N_A   \label{eq:dEdeps} 
\end{equation}

\textcolor{red}{If we make the assumption that the atomic-site electronegativity value is the same for all atoms (A, B, C, $\cdots$),} 
\begin{equation}
\left( \frac{\partial E} { \partial N_A } \right)  =  \left( \frac{\partial E} { \partial N_B } \right)   = \left( \frac{\partial E} { \partial N_C } \right) = \cdots = \mu 
\end{equation} 
 the energy derivative in Eq.~\ref{eq:dEdeps} becomes,
\begin{equation}
\frac { \partial E} { \partial \varepsilon}  = \sum_A  \left( \frac{\partial E} { \partial N_A } \right)  ~ \delta N_A = \mu  \left[ \sum_A   \delta N_A \right] = 0
\end{equation}
which might correspond to Eq. 1.9 in Paul's note. 

\textcolor{red}{This is perplexing because this energy derivative seems to be non-vanishing in some cases.}    
For example, when the external potential is changed by applying an electric field in the x-direction, 
the total energy might be linearly dependent on the value of $\varepsilon$ with a slope proportional to the relaxed dipole moment of the molecule in the x-direction. 

%\bibliographystyle{acs}




\end{document}
