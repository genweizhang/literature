
\documentclass[aip,jcp,preprint,superscriptaddress,amsmath,amssymb]{revtex4-1}
%\documentclass[journal=jpcafh,manuscript=article,layout=twocolumn]{achemso}

\usepackage{fullpage}
\usepackage{graphicx}% Include figure files
\usepackage{mathrsfs}
\usepackage{dcolumn}% Align table columns on decimal point
\usepackage{bm}% bold math
\usepackage{rotating}
\usepackage{textcomp}
\usepackage{float}
\usepackage{subfig}
\usepackage{color}
\usepackage{gensymb}
\usepackage{xspace} % fcp: useful for text macros
\usepackage[normalem]{ulem} % fcp: strike-out text for fcprev

\newcommand{\yerev}[1]{\textcolor{red}{#1}}
\newcommand{\fcprev}[1]{\textcolor{red}{#1}}
\newcommand{\ie}[0]{\textit{i.e.},\xspace} 
\newcommand{\etal}[0]{\textit{et al.}\xspace} 

\begin{document}
\title{Implementation of Periodic Boundary Condition Quantum Mechanical 
Molecular Mechanical Calculations Using the Q-Chem and NAMD Software Packages}
\author{Xiaoliang Pan and Yihan}
\maketitle
%title{Whatever} 

The total QM/MM interaction for a periodic system is
\begin{eqnarray}
E & = &  \frac{1}{2} \left< \rho_{QM}  + q_{MM} \middle| \hat{j}_ {\mathbf{n}} \middle| \rho_{QM}  + q_{MM} \right>  \nonumber \\
& = &  E_{QM,QM} + E_{QM,MM} + E_{MM.MM} 
\end{eqnarray} 
It three components are
\begin{eqnarray}
E_{QM,QM} & = & \frac{1}{2} \left< \rho_{QM}  \middle| \hat{j}_ {\mathbf{n}} \middle| \rho_{QM}  \right> = \frac{1}{2}   \left< \rho_{QM}  | \phi_{QM, \mathbf{n}} \right> \\
E_{QM,MM} & = & \left< \rho_{QM}  \middle| \hat{j}_ {\mathbf{n}} \middle| q_{MM}  \right> =  \left< \rho_{QM}  \middle|   \phi_{MM, \mathbf{n}} \right>  \\
E_{MM,MM} & = & \frac{1}{2} \left< q_{MM}  \middle| \hat{j}_ {\mathbf{n}} \middle| q_{MM}  \right>
\end{eqnarray}
where the electrostatic potential of the QM (or MM) subsystem and its images are denoted by $ \phi_{QM, \mathbf{n}}$  (or $\phi_{MM, \mathbf{n}} $). 

Following the notation from Giese and York, let us split the Coulomb operator $\hat{j}_ {\mathbf{n}}$ into central-cell and extra contributions: 
\begin{equation}
\hat{j}_ {\mathbf{n}} = \hat{j} + \hat{j}_ {\Delta}
\end{equation}
There are several approximations one can make to the QM-QM interaction energy: 
\begin{itemize}
\item Ewald methods (``QM/MM-Ewald") from Nam, Gao and York, and from Holden, Richard and Herbert.  
\begin{eqnarray}
E_{QM,QM}^{Ewald} & =  &  \frac{1}{2} \left< \rho_{QM}  \middle| \hat{j} \middle| \rho_{QM}  \right>  +  \frac{1}{2}  \left< Q_{QM}  \middle| \hat{j}_ {\Delta} \middle| Q_{QM}   \right>   \nonumber \\
 & = & \frac{1}{2} \left< \rho_{QM}  \middle| \hat{j} \middle| \rho_{QM}  \right>  -   \frac{1}{2} \left< Q_{QM}  \middle| \hat{j} \middle| Q_{QM}  \right>  
 +  \frac{1}{2}  \left< Q_{QM}  \middle| \hat{j}_ {\mathbf{n}} \middle| Q_{QM}   \right> \label{eq:Ewq}  
\end{eqnarray} 
which performs an Ewald sum over Mulliken or  ChElPG charges, and then corrects the QM-QM interactions within the central unit cell.   
This leads to $E_{QM,QM}^{Ewq-Mulliken}$ and $E_{QM,QM}^{Ewq-ChElPG}$ models. 

\item Particle Mesh Ewald methods (``QM/MM-PME-C"), which employs atomic charges to represent QM images,  
\begin{eqnarray}
E_{QM,QM}^{PME-Q} & =  &  \frac{1}{2} \left< \rho_{QM}  \middle| \hat{j} \middle| \rho_{QM}  \right>  + \alpha \left< \rho_{QM}  \middle| \hat{j}_ {\Delta} \middle| Q_{QM}  \right>
+ ( \frac{1}{2} -  \alpha)   \left< Q_{QM}  \middle| \hat{j}_ {\Delta} \middle| Q_{QM}  \right> \label{eq:PMEwq}  
\end{eqnarray} 
Here the second term in Eq.~\ref{eq:PMEwq} can be broken into real-space and reciprocal-space components: 
\begin{eqnarray}
\left< \rho_{QM}  \middle| \hat{j}_ {\Delta} \middle| Q_{QM}  \right>  & =  & \left< \rho_{QM}  \middle| \phi_ {\Delta} (Q_{QM})  \right>  \nonumber   \\
& =  & \left< \rho_{QM}  \middle| \phi_ {\mathbf{n}} (Q_{QM})  - \phi (Q_{QM}) \right>   \nonumber \\
& = & \left< \rho_{QM}  \middle|  \phi_ {\mathbf{n}, real} (Q_{QM}) + \phi_ {\mathbf{n}, recip} (Q_{QM})  - \phi (Q_{QM}) \right>   \nonumber  \\
& = & \left< \rho_{QM}  \middle|  \phi_ {\mathbf{n}, real} (Q_{QM})  - \phi (Q_{QM}) \right>  +  \left< \rho_{QM}  \middle|  \phi_ {\mathbf{n}, recip} (Q_{QM})  \right>  \nonumber \\
& = & - \left< \rho_{QM}  \middle|  \hat{j}_{erf}  \middle| Q_{QM} \right> + \left< \rho_{QM}  \middle|  \phi_ {\mathbf{n}, recip} (Q_{QM}) \right>     \label{eq:density-qm-charge} 
\end{eqnarray}
while the third term, $ \left< Q_{QM}  \middle| \hat{j}_ {\Delta} \middle| Q_{QM}  \right>$, in Eq.~\ref{eq:PMEwq} can be treated the same way as in QM/MM-Ewald. 
Note that, as a PME method, the reciprocal component in Eq.~\ref{eq:density-qm-charge} requires us to spread the reciprocal potential from QM atomic charges onto an atom-centered grid
before it can be integrated together with the electron density.   
This reciprocal component contributes to the Fock matrix not only through the electron density (on the bra side) but also the QM atomic charges (on the ket side), 
and it is unclear how to the latter can be computed efficiently within the reciprocal space.  
One way to completely avoid the ket-side contribution is to fix the QM atomic charges to reference values (i.e. kept constant throughout the simulation),
\begin{eqnarray}
E_{QM,QM}^{QM/MM-FC} & =  &  \frac{1}{2} \left< \rho_{QM}  \middle| \hat{j} \middle| \rho_{QM}  \right>  + \alpha \left< \rho_{QM}  \middle| \hat{j}_ {\Delta} \middle| Q_{QM}^{ref} \right>
+ ( \frac{1}{2} -  \alpha)   \left< Q_{QM}^{ref} \middle| \hat{j}_ {\Delta} \middle| Q_{QM}^{ref} \right>  \nonumber \\
 \label{eq:PMEwq}  
\end{eqnarray} 
where the second term becomes
\begin{eqnarray}
\left< \rho_{QM}  \middle| \hat{j}_ {\Delta} \middle| Q_{QM}^{ref}  \right>  =  - \left< \rho_{QM}  \middle|  \hat{j}_{erf}  \middle| Q_{QM}^{ref} \right> + \left< \rho_{QM}  \middle|  \phi_ {\mathbf{n}, recip} (Q_{QM}^{ref} ) \right>
\end{eqnarray}
When the $\alpha$ coefficient is set to 1, this leads to the ambient-potential QM/MM method (``cEw")  of Giese and York.  
\item PME method using electron density for QM images (``QM/MM-PME-D"), and the energy will be 
\begin{equation}
E_{QM,QM}^{PME-D} = \frac{1}{2} \left< \rho_{QM}  \middle|  \hat{j}_{erfc}  \middle| \rho_{QM} \right> + \frac{1}{2}  \left< \rho_{QM}  \middle|  \phi_ {\mathbf{n}, recip} (\rho_{QM}) \right>   
\end{equation} 
It requires the use of $erfc(\xi r ) / r$ operator instead of $1/r$ in the real-space evaluation of Coulomb integrals, 
as well as a representation for the QM density/potential on both cubic grids (for reciprocal-space calculation) and atomc-centered grids (for integration).  
While all necessary code components can be readily found from QM software packages, such as Q-Chem and QPS, 
and from common MM software packages, they have yet to be assembled together.  
\end{itemize}

\clearpage 

\begin{table}[htp]
\setlength{\tabcolsep}{3pt}
\caption{Different options for handling QM-QM interactions in a QM/MM calculation.}
\begin{tabular}{llll}
\hline \hline 
Method & Energy & Expression & Comments \\
\hline 
Non-PBC & Total & $\frac{1}{2} \left< \rho_{QM}  \middle| \hat{j} \middle| \rho_{QM}  \right> $ \\
 & Real Space & $\frac{1}{2} \left< \rho_{QM}  \middle| \hat{j} \middle| \rho_{QM}  \right> $ \\
 & Reciprocal Space & None \\
 Ewald & Total &  $\frac{1}{2} \left< \rho_{QM}  \middle| \hat{j} \middle| \rho_{QM}  \right>  
 + \frac{1}{2} \left[  \left< Q_{QM}  \middle| \hat{j}_ {\mathbf{n}} \middle| Q_{QM} \right>  
 -  \left< Q_{QM}  \middle| \hat{j} \middle| Q_{QM} \right> \right] $ & ``Ewq" \\
 & Real Space & $\frac{1}{2} \left< \rho_{QM}  \middle| \hat{j} \middle| \rho_{QM}  \right>  -  \frac{1}{2}  \left< Q_{QM}  \middle| \hat{j}_ {erf} \middle| Q_{QM}   \right> $  \\
 & Reciprocal Space & $\frac{1}{2} \left< Q_{QM}  \middle|  \phi_ {\mathbf{n}, recip} (Q_{QM}) \right>$    \\
 Ewald-FQ & Total &  $\frac{1}{2} \left< \rho_{QM}  \middle| \hat{j} \middle| \rho_{QM}  \right>  
 + \frac{1}{2} \left[  \left< Q_{QM}^{ref}  \middle| \hat{j}_ {\mathbf{n}} \middle| Q_{QM}^{ref} \right>  
 -  \left< Q_{QM}^{ref}  \middle| \hat{j} \middle| Q_{QM}^{ref} \right> \right] $ & ``MMEw" \\
 & Real Space & $\frac{1}{2} \left< \rho_{QM}  \middle| \hat{j} \middle| \rho_{QM}  \right>  -  \frac{1}{2}  \left< Q_{QM}^{ref}  \middle| \hat{j}_ {erf} \middle| Q_{QM}^{ref}   \right> $  \\
 & Reciprocal Space & $\frac{1}{2} \left< Q_{QM}^{ref}  \middle|  \phi_ {\mathbf{n}, recip} (Q_{QM}^{ref}) \right>$    \\
 PME-Q & Total & $\frac{1}{2} \left< \rho_{QM}  \middle| \hat{j} \middle| \rho_{QM}  \right>   
  + \alpha \left[ \left< \rho_{QM}  \middle| \hat{j}_ {\mathbf{n}} \middle| Q_{QM} \right>  
 -   \left< \rho_{QM}  \middle| \hat{j} \middle| Q_{QM} \right> \right] $ & \\
 & & $ + \left( \frac{1}{2} - \alpha  \right) \left[  \left< Q_{QM}  \middle| \hat{j}_ {\mathbf{n}} \middle| Q_{QM} \right>  
 -   \left< Q_{QM}  \middle| \hat{j} \middle| Q_{QM} \right> \right]$ \\
 & RealSpace & $\frac{1}{2} \left< \rho_{QM}  \middle| \hat{j} \middle| \rho_{QM}  \right>   - \alpha \left< \rho_{QM}  \middle|  \hat{j}_{erf}  \middle| Q_{QM} \right>
 -\left( \frac{1}{2} - \alpha \right)  \left< Q_{QM}  \middle| \hat{j}_ {erf} \middle| Q_{QM}   \right> $  \\
 & Reciprocal Space & $\alpha \left< \rho_{QM}  \middle|  \phi_ {\mathbf{n}, recip} (Q_{QM}) \right>  +  \left( \frac{1}{2} - \alpha \right) \left< Q_{QM}  \middle|  \phi_ {\mathbf{n}, recip} (Q_{QM}) \right> $  \\
 PME-FQ & Total & $\frac{1}{2} \left< \rho_{QM}  \middle| \hat{j} \middle| \rho_{QM}  \right>   
  + \alpha \left[ \left< \rho_{QM}  \middle| \hat{j}_ {\mathbf{n}} \middle| Q_{QM}^{ref} \right>  
 -   \left< \rho_{QM}  \middle| \hat{j} \middle| Q_{QM}^{ref} \right> \right] $ & ``CEw" \\
 & & $ + \left( \frac{1}{2} - \alpha  \right) \left[  \left< Q_{QM}^{ref} \middle| \hat{j}_ {\mathbf{n}} \middle| Q_{QM}^{ref} \right>  
 -   \left< Q_{QM}^{ref}  \middle| \hat{j} \middle| Q_{QM}^{ref} \right> \right]$ \\
 & Real Space & $\frac{1}{2} \left< \rho_{QM}  \middle| \hat{j} \middle| \rho_{QM}  \right>   - \alpha \left< \rho_{QM}  \middle|  \hat{j}_{erf}  \middle| Q_{QM}^{ref} \right>
 -\left( \frac{1}{2} - \alpha \right)  \left< Q_{QM}^{ref}  \middle| \hat{j}_ {erf} \middle| Q_{QM}^{ref}   \right> $  \\
 & Reciprocal Space & $\alpha \left< \rho_{QM}  \middle|  \phi_ {\mathbf{n}, recip} (Q_{QM}^{ref}) \right>  +
   \left( \frac{1}{2} - \alpha \right) \left< Q_{QM}^{ref}  \middle|  \phi_ {\mathbf{n}, recip} (Q_{QM}^{ref}) \right> $  \\
 PME-D & Total & $ \frac{1}{2} \left< \rho_{QM}  \middle|  \hat{j}_{\mathbf{n}}  \middle| \rho_{QM} \right> $  \\
 & Real Space & $ \frac{1}{2} \left< \rho_{QM}  \middle|  \hat{j}_{erfc}  \middle| \rho_{QM} \right> $ \\
 & Reciprocal Space &  $\frac{1}{2}  \left< \rho_{QM}  \middle|  \phi_ {\mathbf{n}, recip} (\rho_{QM}) \right>$ \\
 \hline \hline
 \end{tabular}
 \end{table} 


\clearpage

\begin{table}[htp]
\setlength{\tabcolsep}{3pt}
\caption{Different options for handling QM-MM interactions in a QM/MM calculation.}
\begin{tabular}{llll}
\hline \hline 
Method & Energy & Expression & Comments \\
\hline 
Cutoff-h & Total & $\left< \rho_{QM}  \middle| \hat{j} h (r_C - r)   \middle| q_{MM}  \right> $  \\
& Real Space & $\left< \rho_{QM}  \middle| \hat{j} h (r_C - r)   \middle| q_{MM}  \right> $  \\
& Reciprocal Space & None \\
Cutoff-s & Total & $\left< \rho_{QM}  \middle| \hat{j} s (r - r_C)   \middle| q_{MM}  \right> $  \\
& Real Space & $\left< \rho_{QM}  \middle| \hat{j} s (r - r_C)   \middle| q_{MM}  \right> $  \\
& Reciprocal Space & None \\ 
Ewald & Total & $ \left< \rho_{QM}  \middle| \hat{j}  h (r_C - r)  \middle| q_{MM}  \right> + \left< Q_{QM}  \middle| \hat{j}_{\mathbf{n}} \middle| q_{MM}  \right>  $ 
 & ``Ewq" \\
&  & $ - \left< Q_{QM}  \middle| \hat{j}  h (r_C - r) \middle| q_{MM}  \right> $  \\
& Real Space &  $ \left< \rho_{QM}  \middle| \hat{j}  h (r_C - r) \middle| q_{MM}  \right>  -   \left< Q_{QM}  \middle| \hat{j}_{erf}  h (r_C - r) \middle| q_{MM}  \right> $ \\
& Reciprocal Space & $ \left< Q_{QM}  \middle|  \phi_ {\mathbf{n}, recip} (q_{MM}) \right>$   \\
Ewald-2 & Total & $ \left< \rho_{QM}  \middle| \hat{j} \middle| q_{MM}  \right> + \left< Q_{QM}  \middle| \hat{j}_{\mathbf{n}} \middle| q_{MM}  \right>  $ & \\
&  & $ - \left< Q_{QM}  \middle| \hat{j}_{erfc}  h (r_C - r) \middle| q_{MM}  \right>  - \left< \rho_{QM}  \middle| \hat{j}_{erf}  h (r_C - r) \middle| q_{MM}  \right> $ \\
& Real Space &  $ \left< \rho_{QM}  \middle| \hat{j}_{erfc}  h (r_C - r) \middle| q_{MM}  \right> $  \\
& Reciprocal Space & $ \left< Q_{QM}  \middle|  \phi_ {\mathbf{n}, recip} (q_{MM}) \right>$   \\
Ewald-s & Total & $ \left< \rho_{QM}  \middle| \hat{j} s (r - r_C)  \middle| q_{MM}  \right> + \left< Q_{QM}  \middle| \hat{j}_{\mathbf{n}} \middle| q_{MM}  \right>  $ & \\
&  & $ - \left< Q_{QM}  \middle| \hat{j} s (r - r_C)  \middle| q_{MM}  \right>   $ \\
& Real Space &  $ \left< \rho_{QM}  \middle| \hat{j} s (r - r_C)  \middle| q_{MM}  \right> $  \\
& & $  + \left< Q_{QM}  \middle| \left[ \hat{j}_{erfc} h (r_C - r) - \hat{j} s (r - r_C) \right]  \middle| q_{MM}  \right> $ \\
& Reciprocal Space & $ \left< Q_{QM}  \middle|  \phi_ {\mathbf{n}, recip} (q_{MM}) \right>$   \\
Ewald-sFQ & Total & $ \left< \rho_{QM}  \middle| \hat{j} s (r - r_C)  \middle| q_{MM}  \right> + \left< Q_{QM}^{ref}  \middle| \hat{j}_{\mathbf{n}} \middle| q_{MM}  \right>  $ &  ``MMEw"\\
&  & $ - \left< Q_{QM}^{ref}  \middle| \hat{j} s (r - r_C)  \middle| q_{MM}  \right>   $  & \\
& Real Space &  $ \left< \rho_{QM}  \middle| \hat{j} s (r - r_C)  \middle| q_{MM}  \right> $  \\
& & $  + \left< Q_{QM}^{ref}  \middle| \left[ \hat{j}_{erfc} h (r_C - r) - \hat{j} s (r - r_C) \right]  \middle| q_{MM}  \right> $ \\
& Reciprocal Space & $ \left< Q_{QM}^{ref}  \middle|  \phi_ {\mathbf{n}, recip} (q_{MM}) \right>$   \\
PME & Total & $ \left< \rho_{QM}  \middle| \hat{j}_{\mathbf{n}} \middle| q_{MM}  \right> $ &  ``CEw" \\
& Real Space & $ \left< \rho_{QM}  \middle| \hat{j}_{erfc}  h (r_C - r)  \middle| q_{MM}  \right> $ & \\
& Reciprocal Space &  $\left< \rho_{QM}  \middle|  \phi_ {\mathbf{n}, recip} (q_{MM}) \right>$ & \\
\hline \hline
\end{tabular}
\end{table}




%\bibliographystyle{acs}




\end{document}
